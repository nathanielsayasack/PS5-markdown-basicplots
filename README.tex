\documentclass[]{article}
\usepackage{lmodern}
\usepackage{amssymb,amsmath}
\usepackage{ifxetex,ifluatex}
\usepackage{fixltx2e} % provides \textsubscript
\ifnum 0\ifxetex 1\fi\ifluatex 1\fi=0 % if pdftex
  \usepackage[T1]{fontenc}
  \usepackage[utf8]{inputenc}
\else % if luatex or xelatex
  \ifxetex
    \usepackage{mathspec}
  \else
    \usepackage{fontspec}
  \fi
  \defaultfontfeatures{Ligatures=TeX,Scale=MatchLowercase}
\fi
% use upquote if available, for straight quotes in verbatim environments
\IfFileExists{upquote.sty}{\usepackage{upquote}}{}
% use microtype if available
\IfFileExists{microtype.sty}{%
\usepackage[]{microtype}
\UseMicrotypeSet[protrusion]{basicmath} % disable protrusion for tt fonts
}{}
\PassOptionsToPackage{hyphens}{url} % url is loaded by hyperref
\usepackage[unicode=true]{hyperref}
\hypersetup{
            pdfborder={0 0 0},
            breaklinks=true}
\urlstyle{same}  % don't use monospace font for urls
\usepackage[margin=1in]{geometry}
\usepackage{graphicx,grffile}
\makeatletter
\def\maxwidth{\ifdim\Gin@nat@width>\linewidth\linewidth\else\Gin@nat@width\fi}
\def\maxheight{\ifdim\Gin@nat@height>\textheight\textheight\else\Gin@nat@height\fi}
\makeatother
% Scale images if necessary, so that they will not overflow the page
% margins by default, and it is still possible to overwrite the defaults
% using explicit options in \includegraphics[width, height, ...]{}
\setkeys{Gin}{width=\maxwidth,height=\maxheight,keepaspectratio}
\IfFileExists{parskip.sty}{%
\usepackage{parskip}
}{% else
\setlength{\parindent}{0pt}
\setlength{\parskip}{6pt plus 2pt minus 1pt}
}
\setlength{\emergencystretch}{3em}  % prevent overfull lines
\providecommand{\tightlist}{%
  \setlength{\itemsep}{0pt}\setlength{\parskip}{0pt}}
\setcounter{secnumdepth}{0}
% Redefines (sub)paragraphs to behave more like sections
\ifx\paragraph\undefined\else
\let\oldparagraph\paragraph
\renewcommand{\paragraph}[1]{\oldparagraph{#1}\mbox{}}
\fi
\ifx\subparagraph\undefined\else
\let\oldsubparagraph\subparagraph
\renewcommand{\subparagraph}[1]{\oldsubparagraph{#1}\mbox{}}
\fi

% set default figure placement to htbp
\makeatletter
\def\fps@figure{htbp}
\makeatother


\author{}
\date{\vspace{-2.5em}}

\begin{document}

\section{PS5-markdown-basicplots}\label{ps5-markdown-basicplots}

library(tidyverse) library(dplyr) \#\# Part 1: Load and check data 1.
(1pt) For solving the problems, and answering the questions, create a
new rmarkdown document with an appropriate title. created a newmarkdown
file 2. (2pt) Load data. How many rows/columns do we have? gapminder
\textless{}- read\_delim(``gapminder.csv'') dim(gapminder) 3. (2pt)
Print a small sample of data. Does it look OK? gapminder
\%\textgreater{}\% sample\_n(9) \#\# Part 2: Descriptive Statistics 1.
(3pt) How many countries are there in the dataset? Analyze all three:
iso3, iso2 and name gapminder \%\textgreater{}\% group\_by(name)
\%\textgreater{}\%\\
filter(!is.na(name)) \%\textgreater{}\% summarise(n = n\_distinct(name))
\%\textgreater{}\% arrange(desc(n)) There are 249 unique country names
gapminder \%\textgreater{}\% group\_by(iso2) \%\textgreater{}\%\\
filter(!is.na(iso2)) \%\textgreater{}\% summarise(n=n\_distinct(iso2))
\%\textgreater{}\% arrange(desc(n)) There are 248 unique country iso2
codes gapminder \%\textgreater{}\% group\_by(iso3) \%\textgreater{}\%\\
filter(!is.na(iso3)) \%\textgreater{}\% summarise(n=n\_distinct(iso3))
\%\textgreater{}\% arrange(desc(n)) There are 253 unique country iso3
codes

\begin{enumerate}
\def\labelenumi{\arabic{enumi}.}
\setcounter{enumi}{1}
\tightlist
\item
  If you did this correctly, you saw that there are more names than
  iso-2 codes, and there are even more iso3 -codes. What is going on?
  Can you find it out?
\end{enumerate}

\begin{enumerate}
\def\labelenumi{(\alph{enumi})}
\item
  (5pt) Find how many names are there for each iso-2 code. Are there any
  iso-2 codes that correspond to more than one name? What are these
  countries? gapminder \%\textgreater{}\% group\_by(name)
  \%\textgreater{}\% filter(is.na(iso2)) \%\textgreater{}\% summarize(n
  = n\_distinct(iso2)) \%\textgreater{}\% arrange(desc(n)) Namibia does
  not have an iso2 code and there is another country that does not have
  a country name, just NA
\item
  (5pt) Now repeat the same for name and iso3-code. Are there country
  names that have more than one iso3-code? What are these countries?
  Hint: two of these entitites are CHANISL and NLD CURACAO. gapminder
  \%\textgreater{}\% group\_by(iso3) \%\textgreater{}\%
  filter(is.na(name)) \%\textgreater{}\% summarize(n =
  n\_distinct(name)) \%\textgreater{}\% arrange(desc(n)) CHANISL, GBM,
  KOS, and NLD\_CURACAO have iso3 codes, but they do not have a name
\end{enumerate}

\begin{enumerate}
\def\labelenumi{\arabic{enumi}.}
\setcounter{enumi}{2}
\tightlist
\item
  (2pt) What is the minimum and maximum year in these data?
  min(gapminder\(time, na.rm = TRUE) max(gapminder\)time, na.rm = TRUE)
  1960 is the minimum year; 2019 is the maximum year. \#\# Part 3: CO2
  emissions
\item
  (2pt) How many missing co2 emissions are there for each year? Analyze
  both missing CO2 and co2\_PC. Which years have most missing data?
  gapminder \%\textgreater{}\% filter(is.na(co2\_PC), is.na(co2))
  \%\textgreater{}\% group\_by(time) \%\textgreater{}\% summarise(years
  = n()) \%\textgreater{}\% arrange(desc(years)) 2017, 2018, and 2019
  have the most missing co2 emissions data
\item
  (5pt) Make a plot of total CO2 emissions over time for the U.S, China,
  and India. Add a few more countries of your choice. Explain what do
  you see. gapminder \%\textgreater{}\% filter(iso3
  ==c(``USA'',``CHN'',``IND'',``VNM'', ``UKR'')) \%\textgreater{}\%
  ggplot() + geom\_line(aes(x=time,y=co2,col=iso3)) +
  geom\_point(aes(x=time,y=co2,col=iso3)) I notice that countries that
  are more developed have more CO2 emissions. Countries like Ukraine,
  Vietnam, and India have less CO2 emissions when compared to the US and
  China.
\item
  (5pt) Now let's analyze the CO2 emissions per capita (co2\_PC). Make a
  similar plot of the same countries. What does this figure suggest?
  gapminder \%\textgreater{}\% filter(iso3
  ==c(``USA'',``CHN'',``IND'',``VNM'', ``UKR'')) \%\textgreater{}\%
  ggplot() + geom\_line(aes(x=time,y=co2\_PC,col=iso3)) +
  geom\_point(aes(x=time,y=co2\_PC,col=iso3)) This chart indicates that
  the United States is by far the top in CO2 emissions per capita. All
  other countries are at the bottom of the graph. This might be due to
  the fact that the United States is substantially more developed than
  the other countries listed.
\item
  (6pt) Compute average CO2 emissions per capita across the continents
  (assume region is the same as continent). Comment what do you see
  Note: just compute averages over countries and ignore the fact that
  countries are of different size. Hint: Americas 2016 should be 4.80.
  The average CO2 emissions per capita in the year 2016. gapminder
  \%\textgreater{}\% group\_by(region) \%\textgreater{}\%
  filter(!is.na(co2\_PC), !is.na(region), time == 2016)
  \%\textgreater{}\% summarise(mean = mean(co2\_PC)) \%\textgreater{}\%
  arrange(desc(mean))
\end{enumerate}

The average CO2 emissions per capita for all time. gapminder
\%\textgreater{}\% group\_by(region) \%\textgreater{}\%
filter(!is.na(co2\_PC), !is.na(region)) \%\textgreater{}\%
summarise(mean = mean(co2\_PC)) \%\textgreater{}\% arrange(desc(mean))

\begin{enumerate}
\def\labelenumi{\arabic{enumi}.}
\setcounter{enumi}{4}
\item
  (7pt) Make a barplot where you show the previous results--average CO2
  emissions per capita across continents in 1960 and 2016. gapminder
  \%\textgreater{}\% filter(time \%in\% c(1960, 2016), !is.na(region),
  !is.na(co2\_PC)) \%\textgreater{}\% group\_by(time, region)
  \%\textgreater{}\% summarise(avg\_co2PC = mean(co2\_PC))
  \%\textgreater{}\% ggplot(aes(x=region,y=avg\_co2PC,
  fill=as.factor(time))) + geom\_col(position = ``dodge'') + labs(title
  = ``Average CO2 emissions per capita by year and region'',
  x=``region'', y=``average co2 emissions per capita'') +
  scale\_fill\_discrete(name= ``Year'')
\item
  Which countries are the three largest, and three smallest CO2 emitters
  (in terms of CO2 per capita) in 2019 for each continent? (Assume
  region is continent). The 2019 data is missing, I will use the 2016
  data instead 3 smallest gapminder \%\textgreater{}\% filter(time ==
  ``2016'', !is.na(region), !is.na(co2\_PC)) \%\textgreater{}\%
  group\_by(region) \%\textgreater{}\% arrange(co2\_PC)
  \%\textgreater{}\% slice\_head(n=3) \%\textgreater{}\% select(region,
  name, co2\_PC) 3 largest gapminder \%\textgreater{}\% filter(time ==
  ``2016'', !is.na(region), !is.na(co2\_PC)) \%\textgreater{}\%
  group\_by(region) \%\textgreater{}\% arrange(co2\_PC)
  \%\textgreater{}\% slice\_tail(n=3) \%\textgreater{}\% select(region,
  name, co2\_PC) \#\# Part 4: GDP per capita
\item
  (8pt) Make a scatterplot of GDP per capita versus life expectancy by
  country, using data for 1960. Make the point size dependent on the
  country size, and color those according to the continent. Feel free to
  adjust the plot in other ways to make it better. gapminder
  \%\textgreater{}\% filter(time == 1960, !is.na(region))
  \%\textgreater{}\%
  ggplot(aes(x=GDP\_PC,y=lifeExpectancy,col=region,size=totalPopulation))
  + geom\_point()
\item
  (4pt) Make a similar plot, but this time use 2019 data only. gapminder
  \%\textgreater{}\% filter(time == 2019, !is.na(region))
  \%\textgreater{}\%
  ggplot(aes(x=GDP\_PC,y=lifeExpectancy,col=region,size=totalPopulation))
  + geom\_point()
\item
  (6pt) Compare these two plots and comment what do you see. How has
  world developed through the last 60 years? During the last 60 years,
  all countries' GDP per capita and life expectancy have grown. These
  factors tend to be linked to some extent, since a higher GDP per
  capita corresponds to a longer life expectancy. Yet, the rewards on
  GDP per capita and life expectancy are falling.
\item
  (6pt) Compute the average life expectancy for each continent in 1960
  and 2019. Do the results fit with what do you see on the figures?
  gapminder \%\textgreater{}\% filter(!is.na(region),
  !is.na(lifeExpectancy), !is.na(time)) \%\textgreater{}\% filter(time
  \%in\% c(``1960'',``2019'')) \%\textgreater{}\% group\_by(region,
  time) \%\textgreater{}\% summarize(average = mean(lifeExpectancy))
  Yeah, the findings in terms of life expectancy typically agree with
  the nations from the regions.
\item
  (8pt) Compute the average LE growth from 1960-2019 across the
  continents. Show the results in the order of growth. Explain what do
  you see. gapminder \%\textgreater{}\% filter(!is.na(region),
  !is.na(lifeExpectancy), !is.na(time)) \%\textgreater{}\% filter(time
  \%in\% c(``1960'',``2019'')) \%\textgreater{}\% group\_by(region,
  time) \%\textgreater{}\% summarize(average = mean(lifeExpectancy))
  \%\textgreater{}\% mutate(prev = lag(average), growth = average -
  prev) \%\textgreater{}\% filter(!is.na(growth)) \%\textgreater{}\%
  arrange(desc(growth)) Continents with a large number of developing
  nations have experienced a bigger increase in life expectancy, whereas
  continents with a large number of industrialized countries have seen a
  smaller increase in life expectancy. Several African and Asian
  countries, for example, are still underdeveloped, contributing to the
  region's high growth in life expectancy, whereas European countries
  have been developed for a long time, as seen by their low increase in
  life expectancy.
\item
  (6pt) Show the histogram of GDP per capita for years of 1960 and 2019.
  Try to put both histograms on the same graph, see how well you can do
  it! gapminder \%\textgreater{}\% filter(time \%in\% c(1960, 2019),
  !is.na(GDP\_PC)) \%\textgreater{}\% ggplot(aes(x=GDP\_PC, fill =
  factor(time))) + geom\_histogram(alpha = 0.5, position = ``dodge'',
  bins = 30) + scale\_fill\_manual(values = c(``red'', ``blue''), labels
  = c(``1960'' , ``2019'')) + labs(x = ``GDP per capita'', y= ``Count'',
  title = ``GDP per capita for 1960 and 2019'')
\item
  (6pt) What was the ranking of US in terms of life expectancy in 1960
  and in 2019? (When counting from top.) Hint: check out the function
  rank()! Hint2: 17 for 1960.
\end{enumerate}

1960 life expectancy rank gapminder \%\textgreater{}\% filter(time ==
``1960'', !is.na(lifeExpectancy), !is.na(region)) \%\textgreater{}\%
mutate(rank = rank(-lifeExpectancy)) \%\textgreater{}\% filter(name ==
``United States of America'') \%\textgreater{}\% select(rank) 2019 life
expectancy rank gapminder \%\textgreater{}\% filter(time == ``2019'',
!is.na(lifeExpectancy), !is.na(region)) \%\textgreater{}\% mutate(rank =
rank(-lifeExpectancy)) \%\textgreater{}\% filter(name == ``United States
of America'') \%\textgreater{}\% select(rank) 8. (6pt) If you did this
correctly, then you noticed that US ranking has been falling quite a
bit. But we also have more countries in 2019--what about the relative
rank divided by the corresponding number of countries that have LE data
in the corresponding year? Hint: 0.0904 for 1960.

1960 relative rank gapminder \%\textgreater{}\%
filter(!is.na(lifeExpectancy), time == ``1960'', !is.na(region))
\%\textgreater{}\% mutate(rank = rank(-lifeExpectancy), number\_country
= n(), relativerank = rank/number\_country) \%\textgreater{}\%
select(name, rank, number\_country, relativerank) \%\textgreater{}\%
filter(name == ``United States of America'') \%\textgreater{}\%
select(relativerank)

2019 relative rank gapminder \%\textgreater{}\%
filter(!is.na(lifeExpectancy), time == ``2019'', !is.na(region))
\%\textgreater{}\% mutate(rank = rank(-lifeExpectancy), number\_country
= n(), relativerank = rank/number\_country) \%\textgreater{}\%
select(name, rank, number\_country, relativerank) \%\textgreater{}\%
filter(name == ``United States of America'') \%\textgreater{}\%
select(relativerank)

Finally tell us how many hours did you spend on this PS. I spent over 12
hours on this problem set

\end{document}
