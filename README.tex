\documentclass[]{article}
\usepackage{lmodern}
\usepackage{amssymb,amsmath}
\usepackage{ifxetex,ifluatex}
\usepackage{fixltx2e} % provides \textsubscript
\ifnum 0\ifxetex 1\fi\ifluatex 1\fi=0 % if pdftex
  \usepackage[T1]{fontenc}
  \usepackage[utf8]{inputenc}
\else % if luatex or xelatex
  \ifxetex
    \usepackage{mathspec}
  \else
    \usepackage{fontspec}
  \fi
  \defaultfontfeatures{Ligatures=TeX,Scale=MatchLowercase}
\fi
% use upquote if available, for straight quotes in verbatim environments
\IfFileExists{upquote.sty}{\usepackage{upquote}}{}
% use microtype if available
\IfFileExists{microtype.sty}{%
\usepackage[]{microtype}
\UseMicrotypeSet[protrusion]{basicmath} % disable protrusion for tt fonts
}{}
\PassOptionsToPackage{hyphens}{url} % url is loaded by hyperref
\usepackage[unicode=true]{hyperref}
\hypersetup{
            pdfborder={0 0 0},
            breaklinks=true}
\urlstyle{same}  % don't use monospace font for urls
\usepackage[margin=1in]{geometry}
\usepackage{longtable,booktabs}
% Fix footnotes in tables (requires footnote package)
\IfFileExists{footnote.sty}{\usepackage{footnote}\makesavenoteenv{long table}}{}
\usepackage{graphicx,grffile}
\makeatletter
\def\maxwidth{\ifdim\Gin@nat@width>\linewidth\linewidth\else\Gin@nat@width\fi}
\def\maxheight{\ifdim\Gin@nat@height>\textheight\textheight\else\Gin@nat@height\fi}
\makeatother
% Scale images if necessary, so that they will not overflow the page
% margins by default, and it is still possible to overwrite the defaults
% using explicit options in \includegraphics[width, height, ...]{}
\setkeys{Gin}{width=\maxwidth,height=\maxheight,keepaspectratio}
\IfFileExists{parskip.sty}{%
\usepackage{parskip}
}{% else
\setlength{\parindent}{0pt}
\setlength{\parskip}{6pt plus 2pt minus 1pt}
}
\setlength{\emergencystretch}{3em}  % prevent overfull lines
\providecommand{\tightlist}{%
  \setlength{\itemsep}{0pt}\setlength{\parskip}{0pt}}
\setcounter{secnumdepth}{0}
% Redefines (sub)paragraphs to behave more like sections
\ifx\paragraph\undefined\else
\let\oldparagraph\paragraph
\renewcommand{\paragraph}[1]{\oldparagraph{#1}\mbox{}}
\fi
\ifx\subparagraph\undefined\else
\let\oldsubparagraph\subparagraph
\renewcommand{\subparagraph}[1]{\oldsubparagraph{#1}\mbox{}}
\fi

% set default figure placement to htbp
\makeatletter
\def\fps@figure{htbp}
\makeatother


\author{}
\date{\vspace{-2.5em}}

\begin{document}

\begin{longtable}[]{@{}l@{}}
\toprule
output: html\_document\tabularnewline
\bottomrule
\end{longtable}

\begin{verbatim}
library(tidyverse)
library(dplyr)
\end{verbatim}

\subsection{Part 1: Load and check
data}\label{part-1-load-and-check-data}

\begin{enumerate}
\def\labelenumi{\arabic{enumi}.}
\item
  (1pt) For solving the problems, and answering the questions, create a
  new rmarkdown document with an appropriate title. created a
  newmarkdown file
\item
  (2pt) Load data. How many rows/columns do we have?

\begin{verbatim}
gapminder <- read_delim("gapminder.csv")
dim(gapminder)
\end{verbatim}
\item
  (2pt) Print a small sample of data. Does it look OK?

\begin{verbatim}
gapminder %>% 
  sample_n(9)
\end{verbatim}

  \subsection{Part 2: Descriptive
  Statistics}\label{part-2-descriptive-statistics}
\item
  (3pt) How many countries are there in the dataset? Analyze all three:
  iso3, iso2 and name

\begin{verbatim}
gapminder %>% 
  group_by(name) %>%   
  filter(!is.na(name)) %>% 
  summarise(n = n_distinct(name)) %>% 
  arrange(desc(n))
\end{verbatim}

  There are 249 unique country names

\begin{verbatim}
gapminder %>% 
  group_by(iso2) %>%   
  filter(!is.na(iso2)) %>% 
  summarise(n=n_distinct(iso2)) %>% 
  arrange(desc(n))
\end{verbatim}

  There are 248 unique country iso2 codes

\begin{verbatim}
gapminder %>% 
  group_by(iso3) %>%   
  filter(!is.na(iso3)) %>% 
  summarise(n=n_distinct(iso3)) %>% 
  arrange(desc(n))
\end{verbatim}

  There are 253 unique country iso3 codes
\item
  If you did this correctly, you saw that there are more names than
  iso-2 codes, and there are even more iso3 -codes. What is going on?
  Can you find it out?
\end{enumerate}

\begin{enumerate}
\def\labelenumi{(\alph{enumi})}
\tightlist
\item
  (5pt) Find how many names are there for each iso-2 code. Are there any
  iso-2 codes that correspond to more than one name? What are these
  countries?
\end{enumerate}

\begin{verbatim}
gapminder %>% 
  group_by(name) %>% 
  filter(is.na(iso2)) %>% 
  summarize(n = n_distinct(iso2)) %>% 
  arrange(desc(n))
\end{verbatim}

Namibia does not have an iso2 code and there is another country that
does not have a country name, just NA

\begin{enumerate}
\def\labelenumi{(\alph{enumi})}
\setcounter{enumi}{1}
\item
  (5pt) Now repeat the same for name and iso3-code. Are there country
  names that have more than one iso3-code? What are these countries?
  Hint: two of these entitites are CHANISL and NLD CURACAO.

\begin{verbatim}
gapminder %>% 
  group_by(iso3) %>% 
  filter(is.na(name)) %>% 
  summarize(n = n_distinct(name)) %>% 
  arrange(desc(n))
\end{verbatim}

  CHANISL, GBM, KOS, and NLD\_CURACAO have iso3 codes but they don't
  have a name
\end{enumerate}

\begin{enumerate}
\def\labelenumi{\arabic{enumi}.}
\setcounter{enumi}{2}
\item
  (2pt) What is the minimum and maximum year in these data?

\begin{verbatim}
min(gm$time, na.rm = TRUE)
max(gm$time, na.rm = TRUE)
\end{verbatim}

  1960 is the minimum year and 2019 is the maximum year.
\end{enumerate}

\subsection{Part 3: CO2 emissions}\label{part-3-co2-emissions}

\begin{enumerate}
\def\labelenumi{\arabic{enumi}.}
\tightlist
\item
  (2pt) How many missing co2 emissions are there for each year? Analyze
  both missing CO2 and co2\_PC. Which years have most missing data?
\end{enumerate}

\begin{verbatim}
gapminder %>% 
  filter(is.na(co2_PC), is.na(co2)) %>% 
  group_by(time) %>% 
  summarise(years = n()) %>% 
  arrange(desc(years))
\end{verbatim}

2017, 2018, and, 2019 have the most missing co2 emissions.

\begin{enumerate}
\def\labelenumi{\arabic{enumi}.}
\setcounter{enumi}{1}
\tightlist
\item
  (5pt) Make a plot of total CO2 emissions over time for the U.S, China,
  and India. Add a few more countries of your choice. Explain what do
  you see.
\end{enumerate}

\begin{verbatim}
gm %>% 
  filter(iso3 ==c("USA","CHN","IND","VNM", "UKR")) %>% 
  ggplot() +
  geom_line(aes(x=time,y=co2,col=iso3)) +
  geom_point(aes(x=time,y=co2,col=iso3))
\end{verbatim}

I notice that countries that are more developed have more CO2 emissions.
Countries like Ukraine, Vietnam, and India have less CO2 emissions when
compared to the US and China.

\begin{enumerate}
\def\labelenumi{\arabic{enumi}.}
\setcounter{enumi}{2}
\item
  (5pt) Now let's analyze the CO2 emissions per capita (co2\_PC). Make a
  similar plot of the same countries. What does this figure suggest?

\begin{verbatim}
gm %>% 
  filter(iso3 ==c("USA","CHN","IND","VNM", "UKR")) %>% 
  ggplot() +
  geom_line(aes(x=time,y=co2_PC,col=iso3)) +
  geom_point(aes(x=time,y=co2_PC,col=iso3))
\end{verbatim}

  This chart indicates that the United States is by far the top in CO2
  emissions per capita. All other countries are at the bottom of the
  graph. This might be due to the fact that the United States is
  substantially more developed than the other countries listed.
\item
  (6pt) Compute average CO2 emissions per capita across the continents
  (assume region is the same as continent). Comment what do you see
  Note: just compute averages over countries and ignore the fact that
  countries are of different size. Hint: Americas 2016 should be 4.80.
\end{enumerate}

Here are the average CO2 emissions per capita in the year 2016.

\begin{verbatim}
gapm %>% 
  group_by(region) %>% 
  filter(!is.na(co2_PC), !is.na(region), time == 2016) %>% 
  summarise(mean = mean(co2_PC)) %>% 
  arrange(desc(mean))
\end{verbatim}

The average CO2 emissions per capita for all time

\begin{verbatim}
gapminder %>% 
  group_by(region) %>% 
  filter(!is.na(co2_PC), !is.na(region)) %>% 
  summarise(mean = mean(co2_PC)) %>% 
  arrange(desc(mean))
\end{verbatim}

\begin{enumerate}
\def\labelenumi{\arabic{enumi}.}
\setcounter{enumi}{4}
\tightlist
\item
  (7pt) Make a barplot where you show the previous results--average CO2
  emissions per capita across continents in 1960 and 2016.
\end{enumerate}

\begin{verbatim}
gapminder %>% 
  filter(time %in% c(1960, 2016), !is.na(region), !is.na(co2_PC)) %>% 
  group_by(time, region) %>% 
  summarise(avg_co2PC = mean(co2_PC)) %>% 
  ggplot(aes(x=region,y=avg_co2PC, fill=as.factor(time))) +
  geom_col(position = "dodge") +
    labs(title = "Average CO2 emissions per capita (year and region)",
         x="Region",
         y="Average co2 emissions per capita") +
  scale_fill_discrete(name= "Year")
\end{verbatim}

\begin{enumerate}
\def\labelenumi{\arabic{enumi}.}
\setcounter{enumi}{5}
\tightlist
\item
  Which countries are the three largest, and three smallest CO2 emitters
  (in terms of CO2 per capita) in 2019 for each continent? (Assume
  region is continent).
\end{enumerate}

I used the 2016 DATA

3 smallest

\begin{verbatim}
gapminder %>% 
  filter(time == "2016", !is.na(region), !is.na(co2_PC)) %>% 
  group_by(region) %>% 
  arrange(co2_PC) %>% 
  slice_head(n=3) %>% 
  select(region, name, co2_PC)
\end{verbatim}

3 largest

\begin{verbatim}
gapminder %>% 
  filter(time == "2016", !is.na(region), !is.na(co2_PC)) %>% 
  group_by(region) %>% 
  arrange(co2_PC) %>% 
  slice_tail(n=3) %>% 
  select(region, name, co2_PC) 
\end{verbatim}

\subsection{Part 4: GDP per capita}\label{part-4-gdp-per-capita}

\begin{enumerate}
\def\labelenumi{\arabic{enumi}.}
\tightlist
\item
  (8pt) Make a scatterplot of GDP per capita versus life expectancy by
  country, using data for 1960. Make the point size dependent on the
  country size, and color those according to the continent. Feel free to
  adjust the plot in other ways to make it better.
\end{enumerate}

\begin{verbatim}
gapminder %>% 
  filter(time == 1960, !is.na(region)) %>% 
  ggplot(aes(x=GDP_PC,y=lifeExpectancy,col=region,size=totalPopulation)) +
  geom_point()
  
\end{verbatim}

\begin{enumerate}
\def\labelenumi{\arabic{enumi}.}
\setcounter{enumi}{1}
\item
  (4pt) Make a similar plot, but this time use 2019 data only.

\begin{verbatim}
gapminder %>% 
  filter(time == 2019, !is.na(region)) %>% 
  ggplot(aes(x=GDP_PC,y=lifeExpectancy,col=region,size=totalPopulation)) +
  geom_point()
\end{verbatim}
\item
  (6pt) Compare these two plots and comment what do you see. How has
  world developed through the last 60 years?
\end{enumerate}

During the last 60 years, all countries' GDP per capita and life
expectancy have grown. These factors tend to be linked to some extent,
since a higher GDP per capita corresponds to a longer life expectancy.
Yet, the rewards on GDP per capita and life expectancy are falling.

\begin{enumerate}
\def\labelenumi{\arabic{enumi}.}
\setcounter{enumi}{3}
\tightlist
\item
  (6pt) Compute the average life expectancy for each continent in 1960
  and 2019. Do the results fit with what do you see on the figures?
\end{enumerate}

\begin{verbatim}
gapminder %>% 
  filter(!is.na(region), !is.na(lifeExpectancy), !is.na(time)) %>% 
  filter(time %in% c("1960","2019")) %>% 
  group_by(region, time) %>% 
  summarize(average = mean(lifeExpectancy))
\end{verbatim}

Yeah, the findings in terms of life expectancy typically agree with the
nations from the regions.

\begin{enumerate}
\def\labelenumi{\arabic{enumi}.}
\setcounter{enumi}{4}
\item
  (8pt) Compute the average LE growth from 1960-2019 across the
  continents. Show the results in the order of growth. Explain what do
  you see.

\begin{verbatim}
gapminder %>% 
  filter(!is.na(region), !is.na(lifeExpectancy), !is.na(time)) %>% 
  filter(time %in% c("1960","2019")) %>% 
  group_by(region, time) %>% 
  summarize(average = mean(lifeExpectancy)) %>% 
  mutate(prev = lag(average), growth = average - prev) %>% 
  filter(!is.na(growth)) %>% 
  arrange(desc(growth))
\end{verbatim}

  Continents with a large number of developing nations have experienced
  a bigger increase in life expectancy, whereas continents with a large
  number of industrialized countries have seen a smaller increase in
  life expectancy. Several African and Asian countries, for example, are
  still underdeveloped, contributing to the region's high growth in life
  expectancy, whereas European countries have been developed for a long
  time, as seen by their low increase in life expectancy.
\item
  (6pt) Show the histogram of GDP per capita for years of 1960 and 2019.
  Try to put both histograms on the same graph, see how well you can do
  it!

\begin{verbatim}
gapminder %>% 
  filter(time %in% c(1960, 2019), !is.na(GDP_PC)) %>% 
  ggplot(aes(x=GDP_PC, fill = factor(time))) +
  geom_histogram(alpha = 0.5, position = "dodge", bins = 30) +
  scale_fill_manual(values = c("red", "blue"), labels = c("1960" , "2019")) +
  labs(x = "GDP per capita", y= "Count", title = "GDP per capita for 1960 and 2019")
\end{verbatim}
\item
  (6pt) What was the ranking of US in terms of life expectancy in 1960
  and in 2019? (When counting from top.) Hint: check out the function
  rank()! Hint2: 17 for 1960.
\end{enumerate}

1960 life expectancy rank

\begin{verbatim}
gapminder %>% 
  filter(time == "1960", !is.na(lifeExpectancy), !is.na(region)) %>% 
  mutate(rank = rank(-lifeExpectancy)) %>% 
  filter(name == "United States of America") %>% 
  select(rank)
\end{verbatim}

2019 life expectancy rank

\begin{verbatim}
gapminder %>% 
  filter(time == "2019", !is.na(lifeExpectancy), !is.na(region)) %>% 
  mutate(rank = rank(-lifeExpectancy)) %>% 
  filter(name == "United States of America") %>% 
  select(rank)
\end{verbatim}

\begin{enumerate}
\def\labelenumi{\arabic{enumi}.}
\setcounter{enumi}{7}
\tightlist
\item
  (6pt) If you did this correctly, then you noticed that US ranking has
  been falling quite a bit. But we also have more countries in
  2019--what about the relative rank divided by the corresponding number
  of countries that have LE data in the corresponding year? Hint: 0.0904
  for 1960.
\end{enumerate}

1960 relative rank

\begin{verbatim}
gapmimnder %>% 
  filter(!is.na(lifeExpectancy), time == "1960", !is.na(region)) %>% 
  mutate(rank = rank(-lifeExpectancy), number_country = n(), relativerank = rank/number_country) %>% 
  select(name, rank, number_country, relativerank) %>% 
  filter(name == "United States of America") %>% 
  select(relativerank)
\end{verbatim}

2019 relative rank

\begin{verbatim}
gapminder %>% 
  filter(!is.na(lifeExpectancy), time == "2019", !is.na(region)) %>% 
  mutate(rank = rank(-lifeExpectancy), number_country = n(), relativerank = rank/number_country) %>% 
  select(name, rank, number_country, relativerank) %>% 
  filter(name == "United States of America") %>% 
  select(relativerank)
\end{verbatim}

Finally tell us how many hours did you spend on this PS. I spent 12
hours on this problem set

\end{document}
